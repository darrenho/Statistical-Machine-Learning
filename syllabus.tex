\documentclass[11pt]{article}
\usepackage{multicol}
\usepackage{_defsAndPackages675notation}
\usepackage{amsthm,amssymb}

\topmargin=0in
\headheight=0in
\headsep=0in

\columnsep=-0.28in

\oddsidemargin=0in
\evensidemargin=0in

\textheight=9in
\textwidth=6.5in

\footskip=0in

\begin{document}

\baselineskip=13.2pt
\parindent=0pt
\parskip=13.2pt
\pagestyle{empty}

\centerline{\bf \Large STAT 675 -- Statistical Machine Learning -- Fall 2014}

\begin{center}
\begin{tabular}[h]{llll}
\hline
\hline
\\
\multicolumn{4}{c}{\Large \bf Instructor} \\ 
\\
  & Office & Email & Office Hours \\
Darren Homrighausen~~~~~\,& Stat 204 & {\tt darrenho@stat.colostate.edu} & W 11:00 AM (or Appointment) \\
%&&&  or by Appointment \\
\\
%\multicolumn{4}{l}{All office hours will be held in the UC, 2nd floor of the atrium,
%i.e., above the big-screen TV.} \\
%\\
\hline
\hline
\end{tabular}

\begin{tabular}{lclrcrrr}
{\bf Lecture} & TR & 2:00 PM & -- & 3:15 PM & Statistics 006 \\
\\
{\bf Required Text} & \multicolumn{7}{l}{None (See first lecture for recommended texts)}  \\
\\
{\bf Web Site} & \multicolumn{7}{l}{{\tt Blackboard }} \\ %www.stat.colostate.edu/$\sim$darrenho/SML
\\
{\bf Prerequisites} & \multicolumn{7}{l}{No formal prerequisites.  See summary below for expectations.} \\
\\
{\bf Expectations} & \multicolumn{7}{l}{As this is a 3 credit class, there is a CSU expectation that you spend 6 hours} \\
                              & \multicolumn{7}{l}{outside class on homeworks and review each week.} \\
\\
\hline
\hline
\end{tabular}
\end{center}


%In the classical regime, data sets may have a lot of observations ($n$), but the number ($p$)
%of {\it covariates} or {\it features} was small.  However, there is an increasing
%emphasis on analyzing  {\it high-dimensional} data sets that have $p$ comparable $n$ and often
%exceeding it.  

Upon completing this course, you should be able to tackle modern data analysis problems
by: (1) selecting the appropriate methods and justifying your choices; (2) implementing
these methods programmatically (using, say, the R programming language) and evaluating
your results; (3) having some basis theoretical tools to understand the properties of modern 
machine learning methods and statistical machine learning techniques.


\vspace{.5in}
%\vfill\eject

\newpage

\centerline{{\bf \Large Administrative Remarks}}

\vskip 13.2pt
\centerline{{\it Honor Pledge}} This class operates under the tenets of the CSU honor pledge:
{\it I will not give, receive, or use any unauthorized assistance.}

\vskip 13.2pt
\centerline{{\it Lectures}}
\noindent{\bf Attendance.}  Attendance is mandatory.  You can do it.

\vskip 13.2pt

\centerline{{\it Software}}

\noindent{\bf R.}
In this class you will be provided the opportunity to work with
{\tt R}; a widely used statistical computing platform.  It is free,
moderately well documented, and has a vast user community
(download it from {\tt www.r-project.org}).

\vskip 13.2pt

\centerline{{\it Evaluation}}

\noindent{\bf Scribe duties.} Every lecture, one person will be elected as scribe.  This person's job will be to write up
(using a \LaTeX{} template I will provide) the day's lecture.  There will be topics that I didn't go into much detail
in the lecture but I think would be useful to have more exposition on.  These topics will be notated as e.g.
\[
\textrm{and $\X^{\dagger}$ is the Moore-Penrose pseudo inverse\Note}
\]
The scribe will do the necessary research to provide at least definitions about these concepts and
ideally some useful, related results.  Continuing this example, we might state the four conditions on the pseudo
inverse that make it a Moore-penrose, how to produce it via an SVD, and maybe how it can be used to construct
a projection (ie: $\X \X^{\dagger}$).

The goals of the scribe duties are that it will...
\begin{itemize}
\item encourage critical thinking of the lecture materials
\item produce a nice set of notes for us all to have for reference
\item provide useful \LaTeX{} practice
\item give me some sense of how well the material is being understood
\item provide incentive to collaborate with your fellow students to fill in gaps in your notes/knowledge.
\end{itemize}

\noindent{\bf Homeworks.} There will be periodic homeworks that build on topics from lecture.  These homeworks
will not be formally graded.  However, I'd like to produce nice solutions for each problem.  Hence, I will be collecting
nicely typeset (ie: \LaTeX{}) solutions using a template I provide. I will randomly assign groups of three people to homework teams.  

\noindent{\bf Presentations.} During the semester, there will (hopefully) be topics that arise that you find extraordinarily 
interesting.  I want each person to give two very short ($\leq 10$ minutes) presentations over the course of the semester.
These talks can be informal and can be on anything related to lectures or homeworks.  Possible topics could be a proof
of a result I mention, additional properties of an introduced concept, or some data analysis results using the methods from the class.
%\centerline{{\it Homework and Tests}}
%
%\noindent{\bf Homework.}  
%There will be approximately 7 homework assignments.
%{\it Homework assignments that are turned in late will be assessed a 30\% penalty, 
%regardless of the reason they are late! No homeworks will be accepted more than 48 hours after the due date/time.}  
%Feel free to discuss homework assignments with others, but realize that
%the work you hand in must be your own.
%  \textbf{VERY, VERY IMPORTANT:} For your submitted homeworks, you will be submitting code,
%output, and plots.  Your code must be neat and readable.  Only include output that is directly related to the question you are answering.
%Plots must be labelled (x-axis, y-axis, and title).  Any deviations from this protocal may result in a deduction in points.
%
%\noindent{\bf Tests.}  There will be an in class midterm the last class before Spring Break.  The specifics of the test, such 
%as topics and permitted materials, will be discussed at a later time.
%
%\noindent{\bf Final Policy.} Instead of a final, we will have a final project, the details of which will be determined during the 
%semester.  Each project will be presented in-class May $8^{th}$ and May $10^{th}$ .

\noindent{\bf Exams.} There will be no exams.
\vskip 13.2pt
\centerline{{\it Miscellaneous}}
\noindent{\bf Disability Resources.}
If you require a special accommodation, such as needing
more time to finish exams, contact me \textbf{and} disability services.

%\noindent{\bf Email.}  When sending email, please put ``STAT460" 
%at the beginning of the subject line so that I know the message
%is not spam.  

\newpage



\begin{table}[!h]
\begin{tabular}{lp{6in}}
\multicolumn{2}{c}{{\bf \Large Class Topics}} \\
\hline
\hline
                          & Applied Topics \\
\hline
\\
 &   \textbf{Preliminary materials} \\
 &   \textbf{Linear regression methods:}  \\
           & i. Model selection \\
           & ii. Regularization \\
           & iii. Squared-error risk estimation \\
           & iv. Compressed sensing \\`
           & v. Sparse additive models \\
 &   \textbf{Classification:}  \\
           & i. Discriminant analysis \\
           & ii. Support vector machines \\
           & iii. Kernel methods (Mercer kernels, Reproducing Kernel Hilbert Spaces)\\
           & iv. Tree methods (along with boosting/bagging/bootstrap)\\
           & v. Neural networks \\
          &  {\scriptsize (Note: In most cases, methods can be used for either classification or regression with minor
                         changes)}\\
 &  \textbf{Dimension reduction:}  \\
           & i. Principal Components \\
           & ii. Nonlinear embeddings \\ 
 &   \textbf{Clustering:} \\
           & i. $K$-means \\
           & ii. Hierarchical clustering \\           
\hline
                          & Theoretical Topics \\           
\hline                          
 &  \textbf{Concentration of measure:} \\
           & i. Moment Bounds versus MGF bounds. Hoefdings, McDiarmid, Nemirovski, (generic) chaining \\
           & ii. Empirical processes \\
           & iii. Covering numbers \\
           & iv. VC dimension \\
 &   \textbf{Minimax bounds:} \\
          & i. Function spaces and information theory \\
          & II. Normal means  \\
           & iii. Upper bounds \\
           & iv. Lower bounds \\
\\
\textbf{Note:} & This is a rough overview of possible topics, not a schedule. 
The order of the topics, or even whether they will be included at
all, is subject to change.
\end{tabular}
\end{table}



\end{document}
